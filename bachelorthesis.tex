\documentclass[12pt,a4paper]{scrartcl}

\usepackage[sc]{mathpazo}
\usepackage[T1]{fontenc}
\usepackage[onehalfspacing]{setspace}
\usepackage[utf8]{inputenc}
\usepackage[ngerman]{babel}
\usepackage{geometry}
\usepackage{amsmath}
\usepackage{amsfonts}
\usepackage{amssymb}
\usepackage{graphicx}
\usepackage{natbib}
\usepackage[hyphens]{url}
\usepackage[printonlyused, withpage]{acronym}
\usepackage{listings}
\usepackage{color}

\geometry{a4paper, portrait, left=2.5cm, right=2.5cm, top=2cm, bottom=2cm}
\parindent0cm
\author{Malte Modrow}

\definecolor{lightgray}{rgb}{.9,.9,.9}
\definecolor{darkgray}{rgb}{.4,.4,.4}
\definecolor{purple}{rgb}{0.65, 0.12, 0.82}

\lstdefinelanguage{JavaScript}{
  keywords={typeof, new, true, false, catch, function, return, null, catch, switch, var, if, in, while, do, else, case, break},
  keywordstyle=\color{blue}\bfseries,
  ndkeywords={class, export, boolean, throw, implements, import, this},
  ndkeywordstyle=\color{darkgray}\bfseries,
  identifierstyle=\color{black},
  sensitive=false,
  comment=[l]{//},
  morecomment=[s]{/*}{*/},
  commentstyle=\color{purple}\ttfamily,
  stringstyle=\color{red}\ttfamily,
  morestring=[b]',
  morestring=[b]"
}

\lstdefinelanguage{JavaScript2}{
  keywords={typeof, new, true, false, catch, function, return, null, catch, switch, var, if, in, while, do, else, case, break},
  keywordstyle=\color{green}\bfseries,
  ndkeywords={class, export, boolean, throw, implements, import, this},
  ndkeywordstyle=\color{darkgray}\bfseries,
  identifierstyle=\color{black},
  sensitive=false,
  comment=[l]{//},
  morecomment=[s]{/*}{*/},
  commentstyle=\color{purple}\ttfamily,
  stringstyle=\color{red}\ttfamily,
  morestring=[b]',
  morestring=[b]"
}

\lstset{
   backgroundcolor=\color{lightgray},
   extendedchars=true,
   basicstyle=\footnotesize\ttfamily,
   showstringspaces=false,
   showspaces=false,
   numbers=left,
   numberstyle=\footnotesize,
   numbersep=9pt,
   tabsize=4,
   breaklines=true,
   showtabs=false,
   captionpos=b
}


\begin{document}

\section{Einleitung}
\label{einleitung}
\newpage
\section{Windows 8}
\label{sec:windows8}
Seit dem 26. Oktober 2012 ist Microsofts aktuellstes Betriebssystem \glqq Windows 8\grqq\ für die breite Öffentlichkeit verfügbar. Dieses unterscheidet sich grundlegend von seinem Vorgänger \glqq Windows 7\grqq. Microsoft verfolgt mit Windows 8 den Ansatz, das gleiche Betriebssystem sowohl für Desktop- als auch für Tablet-Computer\footnote{Tablet-Computer: kleiner, flacher, tragbarer Computer mit einem Touchscreen. Es wird von nun an, der Einfachheit halber die englische Version \glqq Tablet\grqq\ benutzt.} zu verwenden. Es besitzt nach wie vor den bekannten Desktop von Windows 7. Äußerlich sind hier nur kleine Änderungen vorgenommen worden. So besitzt z.B. der Windows Explorer eine neues, kontextsensitives Ribbon-Menü\footnote{Auch bekannt als Menüband oder Multifunktionsleiste.}. Neu ist hingegen das "Modern User Interface" (Modern UI, früher Metro), eine für Touch-Gesten optimierte Oberfläche. Obwohl die neue Oberfläche für Touch-Gesten optimiert ist, kann sie trotzdem auch mit Maus und Tastatur bedient werden. Es können, im Gegensatz zum Desktop, keine herkömmlichen Programme im Modern UI ausgeführt werden. In der Modern UI Oberfläche laufen nur Programme (Apps), die speziell für Windows 8 entwickelt wurden und über den Microsoft Store (siehe Sektion \ref{subsubsec:store} auf Seite \pageref{subsubsec:store}) erhältlich sind. Es soll nun ein etwas detaillierterer Blick auf die neue Oberfläche und dessen Eigenheiten geworfen werden.
\subsection{Neues Bedienkonzept}
\label{subsec:bedienkonzept}
%gesten charms usw

\subsection{Windows 8 als hybrides System}
\label{subsec:hybrides system}
% touch, vorteile, nachteile
Windows 8 kann sowohl auf herkömmlichen Desktop Computern als auch auf Tablet-Computern eingesetzt werden. 
\subsection{Windows RT}
\label{subsec:winRT}
\subsection{Das Ökosystem}
\label{subsec:ökosystem}
\subsubsection{Microsoft Store}
\label{subsubsec:store}
\subsubsection{xBox}
\label{subsubsec:xbox}
\subsubsection{Windows Phone}
\label{subsubsec:windowsphone}
\newpage
\section{Konzeption der App}
\label{sec:konzeption}
In diesem Kapitel geht es darum, die Konzeption für eine Windows 8 Nachrichten App darzustellen und zu erläutern. Dabei gilt es darzulegen, warum und mit welchem Hintergrund Entscheidungen zu Gunsten der einen oder der anderen Möglichkeit ausfallen. Dazu müssen zunächst die Ziele der App ausgeführt werden. Anschließend muss sich entschieden werden mit welcher Technologie bzw. Programmiersprache gearbeitet werden soll. Zuletzt wird noch ein detaillierterer Blick auf das Herzstück der App geworfen. Das Navigationskonzept. Hierbei ist eine der wichtigsten Fragen: \glqq Wie kann ich auch bei ggf. großen Datenmengen eine übersichtliche, intuitive Struktur schaffen die sich in das Gesamtkonzept von Windows 8 eingliedert?\grqq  

\subsection{Ziel der App}
\label{subsec:zielderapp}
Das Ziel der zu erstellenden App ist, den Inhalt der ZEIT ONLINE Website auf ansprechende Art und Weise in einer Windows 8 Applikation darzustellen. Das Layout und die Darstellung der Inhalte soll sich an das sogenannte \glqq Look and Feel\grqq\ von Windows 8 anpassen und sich daran orientieren. Der Fokus bei den Inhalten liegt auf den Artikeln selbst und den jeweiligen Aufmacher bzw. Teaser Bildern. Das heißt andere Inhalte wie z.B. Bildergalerien, Infografiken, Blogartikel oder Quizze werden von der App nicht erfasst und nicht dargestellt. Hintergrund ist, die App möglichst einfach zu halten da es in der Fragestellung um die Relation zwischen den Aufmacherbildern und dem dahinter liegenden Artikeltext geht. Die App erhebt in dieser Hinsicht keinen Anspruch darauf, die gesamten redaktionellen Inhalte von \mbox{ZEIT ONLINE} darzustellen, sondern versteht sich eher als explorative Applikation im Sinne der Fragestellung.\\
Der User soll die Möglichkeit haben die standardmäßig vorhandenen Artikeltitel auszublenden um so, wenn gewünscht, einen rein visuellen Eindruck der Artikelbilder zu bekommen. Hierzu soll einen Schalter in der Menüleiste am unteren Bildrand geben. Wenn der User sich im Artikel befindet soll er die Möglichkeit haben die Schriftgröße in gewissen Maß selbst zu bestimmen, da die App eventuell auf Monitoren mit verschieden großen Auflösungen ausgeführt werden wird oder die Sehkraft des Users nicht mehr ausreicht um eine normal große Schrift zu erkennen und zu lesen. So wird in den Artikeln ein gewisses Maß an Barrierefreiheit gewährleistet.\\
Das übergeordnete Ziel der App ist es, eine rudimentäre Nachrichten Applikation für Windows 8 zu erstellen. Gleichzeitig soll eine Umgebung geschaffen werden, die es erlaubt Untersuchungen anzustellen, in wie weit es möglich ist allein durch das Betrachten der Aufmacherbilder auf den Inhalt der jeweiligen Artikel zu schließen (siehe Sektion \ref{subsec:rel_bilder_artikel} auf Seite \pageref{subsec:rel_bilder_artikel}). Außerdem soll die App dazu dienen, zu erforschen, wie eine Nachrichten Applikation in der Modern UI Oberfläche von Windows 8 erstellt wird und welche design- als auch funktionstechnischen Vorgaben von Microsofts vorhanden sind. Sprich, wie eine App mit Nachrichteninhalten nach Microsofts Sichtweise auszusehen hat. 

\subsection{Nativ vs. Web}
\label{nativ_vs_web}
Vor dem Entwickeln einer Windows 8 App muss sich entschieden werden mit welcher Technologie bzw. mit welcher Programmiersprache entwickelt werden soll. Die Rede ist hier von einer App die in der Modern UI Oberfläche von Windows 8 läuft und für diese konzipiert ist. Das heißt es handelt sich nicht um eine klassische .NET oder WIN32 Anwendung für Windows. Um eine Modern UI App zu entwickeln, müssen zwei Dinge zwingend vorhanden sein. Zum einen Windows 8 selbst und zum anderen wird die neueste Version von Visual Studio, Visual Studio 11\footnote{visual Studio 2011 war die aktuellste Version beim Erstellen dieser Arbeit.}, benötigt. Visual Studio steht in der Express Version für Windows 8 kostenlos zur Verfügung. Des weiteren muss sich zwischen der nativen Umsetzung und der Implementierung mit Webtechnologien entschieden werden. Es soll zunächst erläutert werden was die beiden Begriffe bedeuten und in welcher Weise und welchem Zusammenhang sie bei der Entwicklung einer Windows 8 Modern UI App üblicherweise gebraucht werden.

\subsubsection{Native App}
\label{subsubsec:nativ}
Der heutige Begriff \glqq native App\grqq\ unterscheidet sich in einigen Punkten von der früheren oder ursprünglichen Verwendung des Begriffs. Früher sprach man von einer nativen App wenn direkt auf die Ressourcen der  Maschine zugegriffen wurde, wie z.B. Maschinencode der direkt von der CPU ausgeführt wird. Heute wird eine App oftmals schon als nativ deklariert, wenn es sich nicht um eine Webapp handelt. Eine sinnige Definition liegt irgendwo zwischen diesen beiden Varianten. Eine App ist dann nativ, wenn sie Geräte- , Betriebssystem- oder Laufzeitumgebungsabhängig ist. Das heisst, sie ist für ein spezielles Gerät entwickelt und kann nur auf diesem ausgeführt werden. Sie kann dabei alle Geräte- oder Betriebssystemspezifischen Funktionen nutzen, es ist jedoch egal wie nah an der Hardware tatsächlich programmiert wurde \citep{OBrian2013}.\\
In Visual Studio können native Windows 8  Apps u.a. mit den Programmiersprachen C++, C\# oder Visual Basic erstellt werden. Für das Design bzw. das Aussehen der App wird die Markupsprache - \ac{xaml} - verwendet. Es gibt zwei Möglichkeiten wie das \ac{xaml} erstellt werden kann. Es kann entweder von Hand geschrieben werden oder man lässt es sich automatisch generieren. Zum automatischen Generieren lassen sich per Drag \& Drop Elemente wie z.B. Buttons und andere Schaltflächen aus einer Werkzeugpalette direkt auf die \glqq App-Leinwand\grqq\ ziehen, sowie verschieben oder nach Belieben anordnen.
%Listing XAML oder Screenshot 


\subsubsection{Webapp}
\label{webapp}
 
\subsubsection{Windows 8 App mit Webtechnologien}
\label{webwin8}

\newpage
\section{Evaluierung}  
\label{sec:evaluierung}
\subsection{Relation Bilder - Artikelinhalt}
\label{subsec:rel_bilder_artikel} 

\newpage
\section*{Abkürzungsverzeichnis}
\label{sec:abkürzungen}
\begin{acronym}[SEPSEP]
	\acro{xaml}[XAML]{Extensible Application Markup Language}
\end{acronym}

\newpage
\begin{singlespace}
	\bibliographystyle{natdin}
	\bibliography{ba_literatur}
\end{singlespace}

\newpage
\begin{lstlisting}[language= JavaScript2,caption=My Javascript Example]
// Ein eindeutiges Element vom bereitgestellten Zeichenfolgenarray abrufen, in dem ein
// Gruppenschluessel und ein Elementtitel enthalten sein sollten.
function resolveItemReference(reference) {
	for (var i = 0; i < groupedItems.length; i++) {
		var item = groupedItems.getAt(i);
		if (item.group.key === reference[0] && item.title === reference[1]) {
			return item;
		}
	}
}  
   
\end{lstlisting}
\end{document}