\documentclass[12pt,a4paper]{scrartcl}

\usepackage[sc]{mathpazo}
\usepackage[T1]{fontenc}
\usepackage[onehalfspacing]{setspace}
\usepackage[utf8]{inputenc}
\usepackage[ngerman]{babel}
\usepackage{geometry}
\usepackage{amsmath}
\usepackage{amsfonts}
\usepackage{amssymb}
\usepackage{graphicx}
\usepackage{natbib}
\usepackage[hyphens]{url}

\geometry{a4paper, portrait, left=2.5cm, right=2.5cm, top=2cm, bottom=2cm}
\parindent0cm
\author{Malte Modrow}

\begin{document}

\section{Einleitung}
\label{einleitung}
\newpage
\section{Windows 8}
\label{sec:windows8}
Seit dem 26. Oktober 2012 ist Microsofts aktuellstes Betriebssystem \glqq Windows 8\grqq\ für die breite Öffentlichkeit verfügbar. Dieses unterscheidet sich grundlegend von seinem Vorgänger \glqq Windows 7\grqq. Microsoft verfolgt mit Windows 8 den Ansatz, das gleiche Betriebssystem sowohl für Desktop- als auch für Tablet-Computer\footnote{Tablet-Computer: kleiner, flacher, tragbarer Computer mit einem Touchscreen. Es wird von nun an, der Einfachheit halber die englische Version \glqq Tablet\grqq\ benutzt.} zu verwenden. Es besitzt nach wie vor den bekannten Desktop von Windows 7. Äußerlich sind hier nur kleine Änderungen vorgenommen worden. So besitzt z.B. der Windows Explorer eine neues, kontextsensitives Ribbon-Menü\footnote{Auch bekannt als Menüband oder Multifunktionsleiste.}. Neu ist hingegen das "Modern User Interface" (Modern UI, früher Metro), eine für Touch-Gesten optimierte Oberfläche. Obwohl die neue Oberfläche für Touch-Gesten optimiert ist, kann sie trotzdem auch mit Maus und Tastatur bedient werden. Es können, im Gegensatz zum Desktop, keine herkömmlichen Programme im Modern UI ausgeführt werden. In der Modern UI Oberfläche laufen nur Programme (Apps), die speziell für Windows 8 entwickelt wurden und über den Microsoft Store (siehe Sektion \ref{subsubsec:store} auf Seite \pageref{subsubsec:store}) erhältlich sind. Es soll nun ein etwas detaillierterer Blick auf die neue Oberfläche und dessen Eigenheiten geworfen werden.
\subsection{Neues Bedienkonzept}
\label{subsec:bedienkonzept}
%gesten charms usw
\subsection{Windows 8 als hybrides System}
\label{subsec:hybrides system}
% touch, vorteile, nachteile
Windows 8 kann sowohl auf herkömmlichen Desktop Computern als auch auf Tablet-Computern eingesetzt werden. 
\subsection{Windows RT}
\label{subsec:winRT}
\subsection{Das Ökosystem}
\label{subsec:ökosystem}
\subsubsection{Microsoft Store}
\label{subsubsec:store}
\subsubsection{xBox}
\label{subsubsec:xbox}
\subsubsection{Windows Phone}
\label{subsubsec:windowsphone}

\newpage
\begin{singlespace}
	%\bibliographystyle{natdin}
	%\bibliography{ba_literatur}
\end{singlespace}
\end{document}