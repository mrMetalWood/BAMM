\begin{lstlisting}[language= XML,caption=Das XML eines Teaserelements, label={lst:knopfxml}]
<block href="http://xml.zeit.de/digital/internet/2013-08/fablab-open-hardware" year="2013" issue="34" ressort="Digital" author="Tilman Baumgärtel" contenttype="article" publication-date="" expires="" date-last-modified="2013-08-14T12:58:40+00:00" date-first-released="2013-08-14T09:57:43.627551+00:00" date-last-published="2013-08-14T12:59:39.691370+00:00" last-semantic-change="2013-08-14T09:56:40.185797+00:00">
	<supertitle>Open Hardware</supertitle>
	<title>Fab Labs, die Maschinen-Bibliotheken</title>
	<text>
		3-D-Drucker, CNC-Fräsen oder Lasercutter - mit solchen Maschinen sollen Bastler in Fab Labs experimentieren. Immer mehr solcher Werkstätten entstehen nun in aller Welt.
	</text>
	<description>
		3-D-Drucker, CNC-Fräsen oder Lasercutter - mit solchen Maschinen sollen Bastler in Fab Labs experimentieren. Immer mehr solcher Werkstätten entstehen nun in aller Welt.
	</description>
	<byline/>
	<image alt="MakerBot Replicator 2" align="left" title="MakerBot Replicator 2" base-id="http://xml.zeit.de/digital/internet/2013-08/makerbot-cebit-hannover/" type="jpg" publication-date="" expires="">
		<bu>
			Ein MakerBot Replicator 2 auf der diesjährigen Cebit in Hannover
		</bu>
		<copyright>© REUTERS/Fabrizio Bensch</copyright>
	</image>
</block>
\end{lstlisting}

